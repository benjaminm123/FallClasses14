\documentclass[a4paper,11pt]{article}
\usepackage{amsmath}
\usepackage{wrapfig}
\usepackage{fancyhdr}
\usepackage{graphicx}
\usepackage{url}
\usepackage{float}
\usepackage{amsmath}
\usepackage{amssymb}
\usepackage[margin=0.8in]{geometry}

% \setlength{\voffset}{-0.5in}
\setlength{\headsep}{5pt}
\newcommand{\suchthat}{\;\ifnum\currentgrouptype=16 \middle\fi|\;}
\newcommand{\answer}{\textbf{Answer : }}
\newcommand{\cd}{\texttt}
\newcommand{\benu}{\begin{enumerate}}
\newcommand{\enu}{\end{enumerate}}


%===========---------================
% Author John H Allard
% CMPE 12, HW #4
% December 10th, 2014
%===========---------================

\begin{document}
   % \vspace*{\stretch{-0.5}}
   \begin{center}
      \Large\textbf{CMPE 12 Homework \#4}\\
      \large\texttt{John Allard} \\
      \small\texttt{November 20th, 2014}
   \end{center}
   % \vspace*{\stretch{0.5}}

%==========================================
%==========================================
%====== Begin Problems, 15 Total ==========
%==========================================
%==========================================

\benu

% ======== PROBLEM #1 ======== %
\item Write the following in MIPS assembly (with each line commented!) HINT: Use your lab knowledge to do it easily.
\answer 
\texttt{
\\ !    int myarray[5]=\{1,3,5,7,9\};
\\ ADDIU V0, ZERO, 1
\\ SW V0, 4(S8)
\\ ADDIU V0, ZERO, 3
\\ SW V0, 8(S8)
\\ ADDIU V0, ZERO, 5
\\ SW V0, 12(S8)
\\ ADDIU V0, ZERO, 7
\\ SW V0, 16(S8)
\\ ADDIU V0, ZERO, 9
\\ SW V0, 20(S8)
\\ SW ZERO, 0(S8)
} \texttt{
\\ \!    for(int i = 0;i<5;i++)
\\ J 0x9D000150
\\ NOP
\\ LW V0, 0(S8)
\\ ADDIU V0, V0, 1
\\ SW V0, 0(S8)
\\ LW V0, 0(S8)
\\ SLTI V0, V0, 5
\\ BNE V0, ZERO, 0x9D00011C
\\ NOP
}\texttt{
\\ !        myarray[i] = (myarray[i] \& 0xf) << 4;
\\ LW V0, 0(S8)
\\ SLL V0, V0, 2
\\ ADDU V0, S8, V0
\\ LW V0, 4(V0)
\\ SLL V0, V0, 4
\\ ANDI V1, V0, 255
\\ LW V0, 0(S8)
\\ SLL V0, V0, 2
\\ ADDU V0, S8, V0
\\ SW V1, 4(V0)
\\ ! \! \}
\\ ADDU SP, S8, ZERO
\\ LW S8, 28(SP)
\\ ADDIU SP, SP, 32
\\ JR RA
\\ NOP }

\item Write the following in MIPS assembly, with each line commented
\answer \\
\texttt{
\\ !int Fibonacci(int n)
\\ \! \{
\\ ADDIU SP, SP, -32
\\ SW RA, 28(SP)
\\ SW S8, 24(SP)
\\ SW S0, 20(SP)
\\ ADDU S8, SP, ZERO
\\ SW A0, 32(S8)
\\ !if ( n == 0 )
\\ LW V0, 32(S8)
\\ BNE V0, ZERO, 0x9D00010C
\\ NOP}
\texttt{
\\ !return 0;
\\ ADDU V0, ZERO, ZERO
\\ J 0x9D000158
\\ NOP
\\ !else if ( n == 1 )
\\ LW V1, 32(S8)
\\ ADDIU V0, ZERO, 1
\\ BNE V1, V0, 0x9D000128
\\ NOP
\\ !return 1;
\\ ADDIU V0, ZERO, 1
\\ J 0x9D000158
\\ NOP }
\texttt{
\\ !else
\\ !return ( Fibonacci(n-1) + Fibonacci(n-2) );
\\ LW V0, 32(S8)
\\ ADDIU V0, V0, -1
\\ ADDU A0, V0, ZERO
\\ JAL Fibonacci
\\ NOP
\\ ADDU S0, V0, ZERO
\\ LW V0, 32(S8)
\\ ADDIU V0, V0, -2
\\ ADDU A0, V0, ZERO
\\ JAL Fibonacci
\\ NOP
\\ ADDU V0, S0, V0
\\ \! \} 
\\ ADDU SP, S8, ZERO
\\ LW RA, 28(SP)
\\ LW S8, 24(SP)
\\ LW S0, 20(SP)
\\ ADDIU SP, SP, 32
\\ JR RA
\\ NOP
}

\item Given that \cd{a} and \cd{b} are both integers where a and b have been assigned the values 6 and 9 respectively, what is the value of each of the following (12) expressions? If \cd{a} or \cd{b} changes then give their new value.
\benu
\item \cd{a | b} \\
\answer \cd{000110 | 001001 = 001111 =} 15
\item \cd{a || b } \\
\answer \cd{6 || 9} = 1 (or any other non-zero value)
\item \cd{a \& b} \\
\answer \cd{000110 \& 001001 = 000000 = 0}
\item \cd{a \&\& b} \\
\answer \cd{6 \&\& 9} = 1 (or any other non-zero value)
\item \cd{!(a+b)} \\
\answer False, because \cd{(a+b)} is true
\item \cd{a \% b} \\
\answer 6, because $\frac{6}{9} = 0 \text{ r } 6$
\item \cd{b / a} \\
\answer \cd{1}
\item \cd{a = b} \\
\answer \cd{a = 9}, value of expression returned is 9
\item \cd{a = b = 5} \\
\answer \cd{b = 5, a = 5}, the value returned is 5
\item \cd{++a + b--} \\
\answer \cd{a = 7, b = 8} Expression returns the value of \cd{7+9 = 16}
\item \cd{a = (++b < 3)? a : b)} \\
\answer \cd{a = 10, b = 10}, Expression returns a value of \cd{10}
\item \cd{a<<=b}
 \enu

 \item Supposed a program contains the two integer variables \cd{x} and \cd{y}, which have values of 3 and 4 respectively. Write C statements that will exchange the values in \cd{x} and \cd{y} so that after the statements are executed, \cd{x} is equal to 4 and \cd{y} is equal to 3.
 \benu
 \item First do this using a temporary variable for storage \\
 \cd{int temp = x;} \\
 \cd{x = y;} \\
 \cd{y = temp;} \\
 \item Now rewrite this routine without using a temporary variable. \\
 \cd{x = x+y} \\
 \cd{y = x-y} \\
 \cd{x = x-y} \\
 \enu

 \item 
 \benu
 \item Convert the following \cd{while} loop into a \cd{for} loop. \\
 \cd{while (condition)} \\
 \quad \cd{loopBody} \\
 \answer \\
 \cd {for(;condition;)} \\
 \quad \cd{loopbody} \\
 \item Convert the follow \cd{for} loop into a while loop \\
 \cd{for(init;confition;reinit)} \\
 \cd{loopbody} \\
 \answer \\
\cd{init;} \\
\cd{while(condition) \{} \\
\qquad \cd{loopBody;} \\
\cd{reinit;} \\
\cd{\}} \\
\enu

\item Provide the output for each of the following code statements. \\
\benu
\item \answer \cd{******}
\item \answer \cd{**********}
\item \answer \cd{*****} (one for each of the odd numbers, 1-10)
\item \answer \cd{*********}
\enu

\item For each of the following items, identify whether the caller fucntion or the callee function performs the actions.
\benu
\item Writing the parameters into the activation record. \\
\answer Caller
\item Writing the return value. \\
\answer Callee
\item Writing the dynamic link. \\
\answer Callee
\item Modifying the value in R5 to point within the callee functions activation record. \\
\answer Caller
\enu

\item TODO

\item Write a C program that computes the pig-latin translation of an english word. \\
\cd{int translate(char str[]) \{ } \\
\cd{    if(str == NULL) // make sure the argument isn't null } \\
\cd{      return 0; } \\
\cd{    int i = 0; } \\
\cd{    while(str[i+1] != }\verb.'\0'.\cd{) // find the end of the c string } \\
\cd{        i++;        } \\
\cd{    char temp = str[i]; // swap the first and last letters;} \\
\cd{    str[i] = str[0];}
\cd{    str[0] = temp;          } \\
\cd{} \\
\cd{    str[++i] = }\verb.'a'.\cd{; // add on the last two letters} \\
\cd{    str[++i] = }\verb.'y'.\cd{;} \\
\cd{    str[++i] =  }\verb.'\0'.\cd{; // set the null character to mark the new end of the string} \\
\cd{\}}

 \enu

\end{document}