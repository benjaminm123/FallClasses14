\documentclass[a4paper,11pt]{article}
\usepackage{amsmath}
\usepackage{fancyhdr}
\usepackage{graphicx}
\usepackage{url}
\usepackage{float}
\usepackage{amsmath}
\usepackage{amssymb}
\usepackage[margin=1in]{geometry}

\setlength{\voffset}{-0.5in}
\setlength{\headsep}{5pt}
\newcommand{\suchthat}{\;\ifnum\currentgrouptype=16 \middle\fi|\;}


%===========---------================
% Author John H Allard
% CMPE 12, Lab #1 Write-up
% October 9th, 2014
%===========---------================


\title{ CMPE 12 Lab Report \# 1}
\author{John Allard \\ TUTOR \\ MW 2:00 - 4:00pm}
\date{October 6th, 2014}

\begin{document}
\maketitle

\section{Overview}
This lab served as an introduction to circuit design from truth tables using logic gates. All circuits were implemented using the free MultiMedia Logic (MML) program, which allows the user to drag, drop, and connect different circuit components. After the circuit is connected it can be simulated to help check for design errors. This lab consisted of 4 section, with each section varying in difficulty and length. 

\section{Part A}
\subsection{Procedure}
This section was a combined introduction to MML and DeMorgan's laws. To start we were instructed to navigate to an MML tutorial on YouTube\footnote{\url{http://www.youtube.com/watch?v=hJq2gECXYWc&noredirect=1}} and build the circuit that is shown in the tutorial. After this, we were instructed to show our understanding of MMl by implemented DeMorgans laws \(A'+B' = (AB)'\) by building a circuit for each side of the aforementioned equation and showing their equivilence both in practice and by truth table. 
\subsection{Results}
To start, we completed an extremely simple circuit on the MML software. This circuit is about as simple as one can get, it is just a binary switch connected to an LED. For the first circuit, the LED is connected directly to the switch. In this case, if the switch is high, the LED is on, if the switch is low, the LED is off. This can be expressed as a truth table as follows : [\textbf{INSERT TT FIGURE HERE}].

\section{Part B}
\subsection{Procedure}
\subsection{Results}

\section{Part C}
\subsection{Procedure}
\subsection{Results}

\section{Part D}
\subsection{Procedure}
\subsection{Results}
\section{Results}


\end{document}